% 设置 biblatex 额外选项
% \PassOptionsToPackage{gbpub=false, gbtype=false}{biblatex}

% 载入 SJTUThesis 模版
 \documentclass[degree=doctor, zihao=-4, language=english, review]{sjtuthesis}
%\documentclass[degree=master, zihao=-4]{sjtuthesis}
% \documentclass[degree=bachelor, openany, oneside]{sjtuthesis}
% \documentclass[degree=course, language=english, openright, twoside]{sjtuthesis}
% 选项
%   degree=[doctor|master|bachelor|course],     % 必选,学位类型
%   language=[chinese|english],                 % 可选(默认:chinese),论文的主要语言
%   bibstyle=[gb7714-2015|gb7714-2015ay|ieee],  % 可选(默认:gb7714-2015),参考文献样式
%   review,                                     % 可选(默认:关闭),盲审模式

% 所有其它可能用到的包都统一放到这里了,可以根据自己的实际添加或者删除。
\usepackage{sjtuthesis}

% 定义图片文件目录与扩展名
\graphicspath{{figure/}}
\DeclareGraphicsExtensions{.pdf,.eps,.png,.jpg,.jpeg}

% 导入参考文献数据库
\addbibresource{bib/thesis.bib}

% 信息录入,必须在导言区进行!
% !TEX root = ../thesis.tex

%TC:ignore

\title{基于深度学习的圆柱型钢壳表面缺陷识别关键技术与应用研究}
\author{郎贤礼}
\studentid{2009100151}
\supervisor{余有龙教授}
% \assisupervisor{某某教授}
\degree{工学博士}
\major{光电信息工程}
\department{某某系}
\coursename{某某课程}
\date{2019年12月17日}
% \fund{国家 973 项目 (No. 2025CB000000) \\ 国家自然科学基金 (No. 81120250000)}
\keywords{hfut, 仪器, 爱国荣校}

\entitle{RESEARCH ON NONDESTRUCTIVE TIRE DEFECT DETECTION AND CLASSIFICATION USING DEEP LEARNING TECHNOLOGY}
\enauthor{Lang Xianli}Xianli
\ensupervisor{Prof. Mou Mou}
% \enassisupervisor{Prof. Uom Uom}
\endegree{Doctor of Engineering}
\enmajor{A Very Important Major}
\endepartment{Depart of XXX}
\endate{Dec. 17th, 2014}
% \enfund{National Basic Research Program of China (Grant No. 2025CB000000) \\
%   National Natural Science Foundation of China (Grant No. 81120250000)}
\enkeywords{HFUT, master thesis, XeTeX/LaTeX template}

%TC:endignore


% 自定义项目标签名称
% \sjtuSetLabel{
%   listfigure = {图\quad 录},
%   listtable  = {表\quad 录}
% }

\begin{document}

% 无编号内容:中英文论文封面、授权页
\maketitle
\makeDeclareOriginality[pdf/originality.pdf]
\makeDeclareAuthorization

% 使用罗马数字对前言编号
\frontmatter

% 摘要
\input{tex/abstract}

% 目录、插图目录、表格目录
\tableofcontents
\listoffigures
\listoftables
\listofalgorithms

% 主要符号、缩略词对照表
\input{tex/nomenclature}

% 使用阿拉伯数字对正文编号
\mainmatter

% 正文内容
\input{tex/intro}
\input{tex/floats}
\input{tex/math_and_citations}
\input{tex/summary}

% 使用英文字母对附录编号
\appendix

% 附录内容,本科学位论文可以用翻译的文献替代。
\input{tex/app_maxwell_equations}
\input{tex/app_flow_chart}

% 文后无编号部分
\backmatter

% 参考资料
\printbibliography[heading=bibintoc]

% 用于盲审的论文需隐去致谢、发表论文、参与项目、申请专利、简历

% 致谢
\input{tex/acknowledgements}

% 发表论文、参与项目、申请专利、简历
% 盲审论文中,发表学术论文及参与科研情况等仅以第几作者注明即可,不要出现作者或他人姓名
\input{tex/publications}
\input{tex/projects}
\input{tex/patents}
\input{tex/resume}

% 中文学士学位论文要求在最后有一个英文大摘要,单独编页码,英文学士学位论文不需要
\input{tex/end_english_abstract}

\end{document}
